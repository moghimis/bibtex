% Choose the journal abbreviation for the journal you are
% submitting to:
% jgrga JOURNAL OF GEOPHYSICAL RESEARCH
% gbc   GLOBAL BIOCHEMICAL CYCLES
% grl   GEOPHYSICAL RESEARCH LETTERS
% pal   PALEOCEANOGRAPHY
% ras   RADIO SCIENCE
% rog   REVIEWS OF GEOPHYSICS
% tec   TECTONICS
% wrr   WATER RESOURCES RESEARCH
% gc    GEOCHEMISTRY, GEOPHYSICS, GEOSYSTEMS
% (If you are submitting to a journal other than jgrga,
% substitute the initials of the journal for "jgrga" below)
%\documentclass[draft,jgrga]{AGUTeX}
%%%%%%%%%%%%%%%%%%%%%%%%%%%%%%%%%%%%%%%%%%%%%%%%%%%%%%%%%%
%%%% optional article formats author might want to use
% To produce a galley version:
% \documentclass[galley,jgrga]{AGUTeX}
% To produce a two columned version:
% \documentclass[jgrga]{AGUTeX}
%%%%%%%%%%%%%%%%%%%%%%%%%%%%%%%%%%%%%%%%%%%%%%%%%%%%%%%%%%%
%oceandynamics
\documentclass[final]{svjour3}
%\usepackage{newclude}
%\documentclass[twocolumn,final]{svjour3}
%%%%%%%%%%%%%%%%%%%%%%%%%%%%%%%%%%%%%%%%%%%%%%
%\usepackage{newclude}
%
%use with \include*{filename}      % will not add page breaks
\smartqed
\usepackage{float}
\usepackage{graphicx}
\usepackage{textcomp}
\usepackage{natbib}
\usepackage{amsmath}
\usepackage{subfig}
\usepackage{nicefrac}
\usepackage{lineno}
\usepackage{rotating}
\usepackage{enumitem}
\usepackage{multirow}
\usepackage[colorlinks = true,
            linkcolor = blue,
            urlcolor  = blue,
            citecolor = blue,
            anchorcolor = blue]{hyperref}


%\journalname{Ocean Dynamics}
\linenumbers
\linespread{2}
\textwidth 16cm
\textheight 23cm
\newcommand{\comments}[1]{}

\begin{document}

\title{ 
On the modeling of wave-enhanced turbulence quantities near-shore.
}

\author{
Saeed Moghimi et al. $^{1}$
}

\institute{\at OSU.
}

\date{Draft: \today  /  Received: date  / Accepted: date}
% The correct dates will be entered by the editor

\maketitle
\begin{abstract}
{
In this research state-of-the-art \ldots
}

\keywords{SWIFT drifters \and  turbulence \and modeling \ wave  
}
\end{abstract}
%%%%%%%%%%%%%%%%%%%%%%%%%%%%%
%%%%%%%%%%%%%%%%%%%%%%%%%%%%%
%%%%%%%%%%%%%%%%%%%%%%%%%%%%%

\small
\setcounter{tocdepth}{5}
\tableofcontents
\listoffigures
\listoftables
\newpage
\normalsize 


\section{discussions}
\label{sec:discussions}

\subsection{Deep water breaking scaling} 

\subsubsection{Story line:}
\begin{enumerate}
  \item General picture of transferring energy from atmosphere to ocean and the
  rule of the waves. How part of energy input to waves does not dissipate right
  away and carried by surface waves and dissipates later somewhere else (this
  paragraph  could go to the paper introduction).
  \item What we are considering here? Only wave effects due to breaking i.e
  white-capping and depth induced breaking are included (no Langmuir circulation
  or non-breaking wave effects).
  \item How we chose the SWIFT's observation locations (Tab.
  \ref{tab:wcap_points}) and map (Fig. \ref{fig:wcap_map})
  \item Surface flux methods (like Craig and banner type model seems to be the
  most widely used methods for deep water white-capping )
  \item There are two possibilities for surface flux methods. In both case
  surface roughness has to be specified.
  \begin{enumerate}
    \item Parametrization of the surface flux of energy using wind friction
    velocity cubed ( $F_k^\mathrm{s}=\alpha^\mathrm{s} \, {u_*^\mathrm{s}}^3$)
      \begin{enumerate}
          \item \cite{Terrayetal96} scaling and \cite{craig1996velocity} way of
          defining $\alpha^\mathrm{s}$ as:
          
          \cite{craig1996velocity} based on wind tunnel measurements and
          \cite{Terrayetal96} proposed below relationship for
          $\alpha^\mathrm{s}=\dfrac{\bar{c}}{u_*}$ , where $\bar{c}$ and $c_p$
          are effective and phase speed of wave and $u_*$ is surface wind
          friction velocity.
          \begin{align}
            \begin{cases}
            \alpha^\mathrm{s} \sim 150 , & \text{if } \dfrac{c_p}{u_*} > 300 \\
            \alpha^\mathrm{s} \sim 0.5\, \dfrac{c_p}{u_*} , & \text{if } \dfrac{c_p}{u_*} < 300 \\
            \end{cases}
          \end{align}
          and $\alpha^\mathrm{s}$=100 is almost like the default for this
          parameter.
          
          \item Assuming \cite{Terrayetal96} scaling for near surface
          dissipation as $\dfrac{\epsilon \, \mathrm{H}_\mathrm{s}}{\alpha^\mathrm{s} \, u_*^3} =
          0.3 \, \left(\dfrac{z^ \prime}{\mathrm{H}_\mathrm{s}}  \right)^{-2}$, we calculated
          $\alpha^\mathrm{s}$
          \begin{enumerate}
              \item Fig: \ref{fig:alpha_from_swift}, scatter plot of
              $\alpha^\mathrm{s}$ based on \cite{Terrayetal96} scaling.
              $\overline{\alpha^\mathrm{s}}\approx$ 600.
          \end{enumerate}
          \item Using least square method on chosen , we found 
          $\alpha^\mathrm{s}$=968 and $\gamma^\mathrm{s}$=-1.1 based on
          available data assuming $\dfrac{\epsilon \, \mathrm{H}_\mathrm{s}}{\alpha^\mathrm{s} \,
          u_*^3} = 0.3 \, \left(\dfrac{z^ \prime}{\mathrm{H}_\mathrm{s}} 
          \right)^{\gamma^\mathrm{s}}$.   

          \item Discuss this differences as point out by
          \cite{feddersen2007direct}, meaning we might need to think of new
          scaling or formulation for near-shore white-capping dissipation.
          \item Compare the results of 3 different $\alpha^\mathrm{s}$ (100,250 and 1000)
          for a specific surface roughness ($\nicefrac{z_0^\mathrm{s}}{
          \mathrm{H}_\mathrm{s}}$=1.0)
                \begin{enumerate}
                    \item Normalized by H$_\mathrm{s}$ scatter plots
                    (Fig: \ref{fig:alpha_non_dimentional_compare})
                    \item Comparison of vertical profile of TKE and $\epsilon$
                    (Fig: \ref{fig:alpha_compare_profile})
                \end{enumerate}
          \item Results of different$\nicefrac{z_0^\mathrm{s}}
          {\mathrm{H}_\mathrm{s}}$= 0.2, 0.6 and 1.0 for $\alpha^\mathrm{s}$ =
          1000 
          \begin{enumerate}
              \item Fig: Comparison of vertical profile of TKE and
                   $\epsilon$ (Fig. \ref{fig:z0s_compare_profile})
              \item Normalized by H$_\mathrm{s}$ scatter plots (Fig.
                  \ref{fig:z0s_compare_scatter})
              \item Discuss $\alpha^\mathrm{s}$ and surface roughness dynamical
                   role and how they are interact.
              \item Greater roughness induce deeper diffusion (penetration) of
                   energy (refer to Figs above)
              \item Fig: presenting regression lines for 2 different
               $\alpha^\mathrm{s}$ and
               $\nicefrac{z_0^\mathrm{s}}{\mathrm{H}_\mathrm{s}}$ how increase
               in  $\alpha^\mathrm{s}$ move the line in parallel 
              \item changing $\nicefrac{z_0^\mathrm{s}}{\mathrm{H}_\mathrm{s}}$
              rotate the line  e.g. we will produce greater dissipation in
              smaller  dissipation range in model side 
          \end{enumerate}
          \item Table of RMSE and BIAS for both TKE and $\epsilon$ (here or
          appendix?)
      \end{enumerate}
      \item In case of access to wave model, using surface dissipation of the 
       waves as source of the energy into the water column.
          \begin{enumerate}
              \item Definition of wave surface dissipation in this context.
              Based on SWAN dissipation $S^{\mathrm{ds}}$ terms as: the dissipation rates
              due to bottom friction $S^{\mathrm{ds,b}}$,
              surface wave breaking $S^{\mathrm{ds,br}}$ and white capping $S^{\mathrm{ds,w}}$
              \citep{booij2004swan}. 
              \item We plugged in $S^{\mathrm{ds,s}}$=$S^{\mathrm{ds}}
              -S^{\mathrm{ds,b}}$ as surface dissipation in to ocean model
              (includes depth induced breaking too in case.
              \item Compare vertical TKE and $\epsilon$ for four cases as: KOM
              and WES (Tab. \ref{tab:wcap_methods})for normal and broad frequency range (Fig.
              \ref{fig:wave_freq_range_compare_profile})
              \item Compare wind input and white-capping dissipation for the same
              set of SWAN cases as above (frequency range $\times$ white-capping
              parametrization) (Fig. \ref{fig:wcap_win_wave}) 
              \item Argue how different white-capping and wind input scheme
              keep total energy in balance, but gives higher energy
              dissipation rates. 
              \item It seems there is still rooms for improvement.
              \item Comparison of the best one for different ???
              $\nicefrac{z_0^\mathrm{s}}{\mathrm{H}_\mathrm{s}}$
          \end{enumerate} 

          \item compare flux of energy from wave model $F_k^\mathrm{s,wave}=g \,
          S^{\mathrm{ds,s}}$ to $F_k^\mathrm{s}=\alpha^\mathrm{s} \,
          {u_*^\mathrm{s}}^3$ using wind velocity (for 2 typical $\alpha^\mathrm{s}$
          like 100 and 250) with default SWAN white capping with different frequency
          range (white-capping in spectrum tail) and the same for new
          parametrization (Figs).   
          \item in fact, if friction velocity cubed from wind and
          from wave model matches. This means with increasing
          $\alpha^\mathrm{s}$, we are multiplying the amount of energy dissipated from
          waves by some factors.
          It might basically point to a lack  in the theory. Can we come up
          with a factor to energy and try energy input with that factor? 
          \item compare  Fig: normalized by H$_\mathrm{s}$ scatter plots  for wave
          dissipated and wind scaled
          \item describe uncertainties for estimation of $\alpha^\mathrm{s}$ and wide range
          of values using available theories.
          \item It seems using wave models is the most physical sound method. Due
          to possibility of order of magnitude errors steak to wave dissipation from
          wave models makes sense.
  \end{enumerate}

  \item distribution of total energy into the water column methods:
      \begin{enumerate}
        \item Convert amount of flux to volume of energy and distribute it as tke
        production terms in water column using below methods:
        \item Describe \cite{sullivan2004oceanic,sullivan2007surface}
        \item Describe \cite{kudryavtsev2008vertical}
        \item Comparing this methods with flux methods (Fig.
        \ref{fig:wave_vertical_prod}).
        \item Advantages (we no longer need to specify surface roughness)
        \item Disadvantages ??? 
       \end{enumerate}
\item Conclude the white-capping energy dissipation methods and usefulness of
SWIFTs for this purpose.

\end{enumerate}

\begin{table}
 \begin{center}
   \caption{Criteria for choosing SWIFT measurement points.}
   \label{tab:wcap_points}
   \begin{tabular}{c c c}
    \hline
        &  White-capping breaking         &  Depth induced breaking   \\ \hline 
        Water Depth      &  $>$ 5 [m]       &    $<$ 2 [m]             \\
        Wind speed       &  $>$ 5.5 [ms-1]       &    0.0 [ms-1]          \\
        $\mathrm{H}_\mathrm{s}/Depth$      &  $<$ 0.1        &    $>$ 0.3               \\ \hline
   \end{tabular}
 \end{center}
\end{table}


\begin{figure}
   \centering
   %\hspace*{-20mm}
   \includegraphics[width=0.75
   \textwidth]{../pics/x1-wcap/craig_banner_whitecapping/alpha/map-wcap.pdf}
   \caption{SWIFT locations for white-capping type breaking measurements (34
   locations and 442 point measurements for TKE and $\epsilon$.}
   \label{fig:wcap_map}
\end{figure}


\begin{figure}
   \centering
   %\hspace*{-20mm}
   \includegraphics[width=0.7
   \textwidth]{../pics/x1-wcap/craig_banner_whitecapping/alpha/alpha.pdf}
   \caption{Calculated $\alpha^\mathrm{s}$ from SWIFT measurements based on
   \cite{Terrayetal96} scaling for near-shore white-capping ( 5[m]$>$ depth $>$
   8[m]).
   }
   \label{fig:alpha_from_swift}
\end{figure}


\begin{figure}
   \centering
   %\hspace*{-20mm}
   \includegraphics[width=0.75
   \textwidth]{../pics/x1-wcap/craig_banner_whitecapping/alpha/02_zp2hs_non_dimen_wave-crop.pdf}
   \caption{Comparing non-dimensionalized TKE and $\epsilon$ by wave
   height for $\alpha^\mathrm{s}=$100, 250 and 1000. }
   \label{fig:alpha_non_dimentional_compare}
\end{figure}


\begin{figure}
   \centering
   %\hspace*{-20mm}
   \includegraphics[width=0.85
   \textwidth]{../pics/x1-wcap/craig_banner_whitecapping/alpha/tke-eps0-crop.pdf} 
   \caption{Vertical profile of TKE and $\epsilon$ for
   $\nicefrac{z_0^\mathrm{s}}{\mathrm{H}_\mathrm{s}}$ =1.0 and
   $\alpha^\mathrm{s}=$ 100, 250 and 1000. left: TKE, right: $\epsilon$ for
   points 238 and 1309 (See Fig. \ref{fig:wcap_map}).
   }  
   \label{fig:alpha_compare_profile}
\end{figure}


\begin{figure}
   \centering
   %\hspace*{-20mm}
   \includegraphics[width=0.85
   \textwidth]{../pics/x1-wcap/craig_banner_whitecapping/z0s/tke-eps0-crop.pdf} 
   \caption{Vertical profile of TKE and $\epsilon$ for
   $\nicefrac{z_0^\mathrm{s}}{\mathrm{H}_\mathrm{s}}$=0.2, 0.4, 0.6, 1.0 and 1.6
   with $\alpha^\mathrm{s}$=1000. left: TKE, right: $\epsilon$ for
   points 238 and 1309 (See Fig. \ref{fig:wcap_map}).}  
   \label{fig:z0s_compare_profile}
\end{figure}

\begin{figure}
   \centering
   %\hspace*{-20mm}
   \includegraphics[width=1.0
   \textwidth]{../pics/x1-wcap/craig_banner_whitecapping/z0s/tke22-crop.pdf} 
   \caption{Scatter plot of TKE and $\epsilon$ for
   $\nicefrac{z_0^\mathrm{s}}{\mathrm{H}_\mathrm{s}}$=0.2, 0.4, 0.6, 1.0 and 1.6
   with $\alpha^\mathrm{s}$=1000. left: TKE, right: $\epsilon$ for
   points 238 and 1309 (See Fig. \ref{fig:wcap_map}).}  
   \label{fig:z0s_compare_scatter}
\end{figure}

\begin{figure}
   \centering
   %\hspace*{-20mm}
   \includegraphics[width=0.85
   \textwidth]{../pics/x1-wcap/swan_whitecapping/tke-eps0-crop.pdf} 
   \caption{Effects of different frequency range and white-capping methods
   in wave model on vertical profile of TKE and $\epsilon$ results from the
   ocean model.   For all cases
   $\nicefrac{z_0^\mathrm{s}}{\mathrm{H}_\mathrm{s}}$=0.2.  left:  TKE, right:
   $\epsilon$ for points 238 and 1309 (See Fig. \ref{fig:wcap_map}).} 
   \label{fig:wave_freq_range_compare_profile}
\end{figure}

\begin{figure}
   \centering
   %\hspace*{-20mm}
   \includegraphics[width=0.9
   \textwidth]{../pics/x1-wcap/swan_whitecapping/wcap_sim-win_sim-hs_sim-crop.pdf} 
   \caption{Comparison of white-capping (a), wind input (b) and Significant wave
   hight (c), for different frequency range, white-capping methods. (See Fig.
   \ref{fig:wcap_map} for locations and Tab. \ref{tab:wcap_methods} for wave
   model settings).}
   \label{fig:wcap_win_wave}
\end{figure}

\begin{table}
 \begin{center}
   \caption{Wave model source and sink terms setting.}
   \label{tab:wcap_methods}
   \begin{tabular}{l c c}
    \hline
          &     Wind input       &  White-capping  \\
         \hline KOM    &   \cite{snyder1981array}  &  
         \cite{komen1984existence}
         \\
                WES \citep{mulligan2008whitecapping}  &   \cite{yan1987improved}    
                &\cite{alves2003performance} \\\hline
   \end{tabular}
 \end{center}
\end{table}

\begin{figure}
   \centering
   %\hspace*{-20mm}
   \includegraphics[width=0.9
   \textwidth]{../pics/x1-wcap/swan_whitecapping/surf_flux2-crop.pdf} 
   \caption{Comparison of surface energy flux to ocean model using wind friction
   velocity cubed $F_k^\mathrm{s,wind}=\alpha^\mathrm{s} \, {u_*^\mathrm{s}}^3$ 
   for $\alpha^\mathrm{s}$=100, 250 and 1000 and surface energy flux using wave
   model surface dissipation  $F_k^\mathrm{s,wave}=g \, S^{\mathrm{ds,s}}$. (See
   Fig. \ref{fig:wcap_map} for locations).}
   \label{fig:surface_flux_compare}
\end{figure}

% \begin{figure}
%    \centering
%    %\hspace*{-20mm}
%    \includegraphics[width=0.7
%    \textwidth]{../pics/x1-wcap/total_energy_methods/tke-eps0-crop.pdf} 
%    \caption{Effects of different frequency range and white-capping methods
%    in wave model on vertical profile of TKE and $\epsilon$ results from the
%    ocean model.   For all cases
%    $\nicefrac{z_0^\mathrm{s}}{\mathrm{H}_\mathrm{s}}$=0.2.  left:  TKE, right:
%    $\epsilon$ for points 238 and 1309 (See Fig. \ref{fig:wcap_map}).} 
%    \label{fig:wave_vertical_prod}
% \end{figure}


\begin{table}
 \begin{center}
   \caption{The surface roughness and $\alpha^\mathrm{s}$ parameter proposed in
   the literature.}
   \label{tab:surface_roughness_alpha_litrature}
     \begin{tabular}{l p{3cm} c p{5cm} }
    %\begin{tabular}{l l p{3.5cm}  }
      \hline
                                    &  ${z_0^\mathrm{s}}$    &  $\alpha^\mathrm{s}$ & specifications\\ \hline 
      
      \cite{craig1994modeling}      &  ${z_0^\mathrm{s}}=0.1 m $            & 100 $\sim$ 150 & using \cite{MellorYamada82} turbulence model
\\
      \cite{GemmrichFarmer99}       &  ${z_0^\mathrm{s}}=0.2 m $            & * & 
      Micro-structure measurements under large waves ($\mathrm{H}_\mathrm{s}=$3.5 m)  
      \\
      \cite{burchard2001simulating} &  $0.2 <  {z_0^\mathrm{s}} / {\mathrm{H}_\mathrm{s}} <1$  & 100 & 
      $k-\epsilon$ turbulence model with  modified Schmidt number   \\
      \cite{Terrayetal99}           &  ${z_0^\mathrm{s}} / {\mathrm{H}_\mathrm{s}}  =0.85$     & 100& 
      \cite{craig1994modeling} with modified length scale
\\
      \cite{Umlauf03a}              &  ${z_0^\mathrm{s}} / {\mathrm{H}_\mathrm{s}}  \simeq 1$   &100 & 
      $k-\omega$ via generic length scale model
   \\
      \cite{kantha2004effect}       &  ${z_0^\mathrm{s}} / {\mathrm{H}_\mathrm{s}} = 1.6 $      &100 &
      Assuming fully developed sea
      
\\
      \cite{stips2005m}             &  ${z_0^\mathrm{s}} / {\mathrm{H}_\mathrm{s}}  \ll  1$    &100&  
      Method of \cite{Umlauf03a} for low wind condition in small lake
\\
      \cite{feddersen2007vertical}     &  Not calculated   & 250 &  
     Based on observations in 3.5m water depth proposing different scaling for
     white capping breaking in near-shore regions.
     \\
      \cite{jones2008influence}     &  ${z_0^\mathrm{s}} / {\mathrm{H}_\mathrm{s}} = 1.3 $     & 60 &  
      $k-\omega$  based on \cite{Umlauf03a} for shallow wind forced environment
      with tide   \\      
      
      \cite{newberger2007forcing2}  &0.2, 0.3, 0.5 [m] and  ${z_0^\mathrm{s}} / {\mathrm{H}_\mathrm{s}} =
      0.15 $ & 100 & $k-kl$ model of \cite{MellorYamada82} for surf-zone 
      \\
      
      \cite{moghimi2013direct}  &${z_0^\mathrm{s}} / {\mathrm{H}_\mathrm{s}} = 0.3 $ & 100 &  $k-\omega$ 
      based on \cite{Umlauf03a} for surf-zone     \\
      
      \hline
     \end{tabular}
 \end{center}
$^{*}$ Instead of the surface flux of the turbulent kinetic energy to be
proportional to the cube of the surface friction velocity with $\alpha^\mathrm{s}$=100,
They used $c_p u_*^2$ where $c_p$= 0.8[ms-1] is the effective phase speed of waves
acquiring energy from the wind (Gemmrich et al. 1994)
\end{table}





\subsection{Shallow water depth induced breaking}
\subsubsection{Story line:}
\begin{enumerate}
    \item Classification based on
    $\gamma_{br}=\nicefrac{\mathrm{H}_\mathrm{s}}{Depth}$
    \item Consistent pattern of $\dfrac{F_k^\mathrm{s,wave}}{g
    S^{\mathrm{ds,s}}} \leq $ 0.2 for diffrent range of $\gamma_{br}$ and number
    of points.
    \item 
\end{enumerate}


\comments{

Modeling challenges:

\begin{enumerate}[label=\alph*)]
  \item Original Craig and Banner method using ${u_*^\mathrm{s,wind}
  }^3$ depends on surface roughness and $\alpha^\mathrm{s}$
  \item Calculating $F_k$ from white-capping (wave model surface dissipation)
  with the same surface roughness shows far less energy
  \item Laking energy due to not considering high frequency tail of the spectrum
  could ruin our simulation
  \item Effects of adding high frequency tail improves the results however still
  off
  \item Different white capping terms might helps. In the new term the wind
  input term and white-capping terms are more or less in balance but we see
  larger values for both terms.
  \item the best we can got for explicit energy input is presented as with this
  method
  \item What if we calculate friction velocity based on dissipated waves and use
  the same eta and check the system again. 
  \item in fact comparing friction velocity cubed from wind and from wave model
  dissipated energy matches. This means with increasing eta we are multiplying
  the amount of energy dissipated from waves. It might basically point to a lack
  in the theory. Can we come up with a factor to energy and try energy input
  with that factor? 
  \item how about total energy methods? Kudritsev and sulivan ? 
\end{enumerate}

Graphs:
\begin{enumerate}
  \item All points : frequency range, white capping methods, craig with same 
  \item 

\end{enumerate}

\subsubsection{Surface flux of energy from wind friction velocity  }
In this type of approach turbulnce model in ocean side remethod calculation of
surface flux and surface roughness includes diffrent constants here we are discusing how these constants were choosen.

A relatively wide range of suggested values for surface roughness length could be found in literature. \cite{stips2005m}
mentioned that the magnitude of $z_0^\mathrm{s}$ depends on the method of observation. For example $z_0^\mathrm{s}=\mathrm{H}_\mathrm{s}$ was reported
from a fixed tower measurement and $z_0^\mathrm{s}=0.2m$ was calculated with a floating instrument in the sea state with
$\mathrm{H}_\mathrm{s}=5m$. On the other hand 
A comparative study for a set of different surface roughness equal to 0.2, 0.3, 0.5 and 0.2 $\mathrm{H}_\mathrm{s}$ is given by
\cite{newberger2007forcing2}.  

A relatively wide range of suggested values for surface
roughness length is available in  literature starting from a few centimeters,
e.g. $ ~ 0.1m$ by CB94, up to values greater than significant wave height, $\mathrm{H}_\mathrm{s}$
(see table \ref{tab:surface_roughness}).  \cite{stips2005m} mentioned that the
magnitude of  $z_0^\mathrm{s}$ depends on the method of observation. For example
$z_0^\mathrm{s}=\mathrm{H}_\mathrm{s}$ was reported from a fixed tower measurement and $z_0^\mathrm{s}=0.2m$ was
calculated with a floating instrument in the sea state with $\mathrm{H}_\mathrm{s}=5m$.  





\subsubsection{$\alpha^\mathrm{s}$ From the litrature:}
\cite{craig1996velocity} based on wind tunnel measurments and
\cite{Terrayetal96} proposed below relationship for $\alpha^\mathrm{s}=\dfrac{\bar{c}}{u_*}$ , where $\bar{c}$ is
effective wave speed and $u_*$ is surface wind friction velocity.
 
\begin{align}
\begin{cases}
\alpha^\mathrm{s} \sim 150 , & \text{if } \dfrac{c_p}{u_*} > 300 \\
\alpha^\mathrm{s} \sim 0.5\, \dfrac{c_p}{u_*} , & \text{if } \dfrac{c_p}{u_*} < 300 \\
\end{cases}
\end{align}

However $\alpha^\mathrm{s}=100$ were reported frequently in the litrature
in diffrent applications ranging from small lakes to open ocean
\citep{burchard2001simulating,Umlauf03a,kantha2004effect,stips2005m}.

\cite{jones2008influence} employed a two-equation $k-\omega$ turbulence model
for shallow wind forced environment with tide and suggetsed $\alpha^\mathrm{s}=60$ with
surface roughness of ${z_0^\mathrm{s}} / {\mathrm{H}_\mathrm{s}} = 1.3$.

\cite{feddersen2007vertical}  evaluate $\alpha^\mathrm{s}$ based on nearshore setting data
gathering in water depth about 3.5 [m]. The applied \cite{Terrayetal96} scaling
for near surface dissipation as:
\begin{align}
\dfrac{\epsilon \, \mathrm{H}_\mathrm{s}}{\alpha^\mathrm{s} \, u_*^3} =
0.3 \, \left(\dfrac{z^ \prime}{\mathrm{H}_\mathrm{s}}  \right)^{-2} , 
\label{eq:terray_scaling}
\end{align}
and suggetsted $\alpha^\mathrm{s}=250$. They argued that perhaps the scaling of surface
energy flux as function of surface friction velocity cubed in shallow nearshore
waters is diffrent from deep ocean.

From SWIFT measurment points we chose the whitecapping breaking following
conditions in Tab. \ref{tab:chosen_points}



The location of the data points chosen for whitecapping type breaking show in
Fig. \ref{fig:wcap_map}.

\begin{figure}
   \centering
   %\hspace*{-20mm}
   \includegraphics[width=0.6
   \textwidth]{../pics/x1-wcap/craig_banner_whitecapping/02_alpha_compare/map-wcap.pdf}
   \caption{SWIFT locations for whitecapping type breaking measurments (34
   locations and 442 measurments for TKE and $\epsilon$.}
   \label{fig:wcap_map}
\end{figure}

We calculated $\alpha^\mathrm{s}$ parameter using SWIFT's dissipation measurments for 34
chosen  points for whitecapping type breaking using equation
\ref{eq:terray_scaling} (Fig. \ref{fig:alpha_from_swift}). It should be noted we
have used simple formula to evalute wind friction velocity \citep{wu1994sea}.
Simple average of this parameter gives $\alpha^\mathrm{s}$=594. This is more than two times
greater than resulted values by \cite{feddersen2007direct}. 

% \begin{figure}
%    \centering
%    %\hspace*{-20mm}
%    \includegraphics[width=0.45
%    \textwidth]{../pics/x1-wcap/craig_banner_whitecapping/01_alpha/alpha-crop.pdf}
%    \caption{Calculated $\alpha^\mathrm{s}$ from SWIFT measurments based on
%    \cite{Terrayetal96} scaling for nearshore whitecapping ( 5[m]$>$ depth $>$
%    8[m]).
%    }
%    \label{fig:alpha_from_swift}
% \end{figure}


% \begin{figure}
%    \centering
%    %\hspace*{-20mm}
%    \includegraphics[width=0.55
%    \textwidth]{../pics/x1-wcap/craig_banner_whitecapping/02_alpha_compare/zp2h-terray-scaling.pdf}
%    \caption{Comparing results for \cite{Terrayetal96} scaling for $\alpha^\mathrm{s}=$100,
%    250 and 600. }
%    \label{fig:alpha_compare}
% \end{figure}
% 
%  
% \begin{figure}
%    \centering
%    %\hspace*{-20mm}
%    \includegraphics[width=0.55
%    \textwidth]{../pics/x1-wcap/craig_banner_whitecapping/03_z03_comparison/tke-eps0-z0s_comp.pdf}
%    \caption{TKE and $\epsilon$ vertical profile for different surface roughness.
%    Only the first two selected location from SWIFT measurments shown here.
%    }
%    \label{fig:z0s_compare}
% \end{figure}
 

So far we come up with the best fit $\alpha^\mathrm{s}$=600 and perhaps surface roughness
$z_0^\mathrm{s}=0.6 \mathrm{H}_\mathrm{s} \sim 1.0 \mathrm{H}_\mathrm{s}$.
Now let see if we replace surafecfulx of TKE with energy dissipated from waves
which should be more physicaly sound way to handle energy tranfwer from
atmospher to ocean.

\subsubsection{Whitecapping dissipation from wave model}
what to discusse here:
\begin{enumerate}[label=\alph*)]
  \item effects of frequency range
  \item effects of whitecapping and wind input methods
  \item comparison of diffrent two-equation model!
\end{enumerate}



capture outer surfzone
deep water breaking (whitecapping type breaking) following below mention
creteria is given in Fig \ref{fig:alpha_from_swift}.

\subsection{Shallow water breaking (inner surf-zone)}
\begin{enumerate}[label=\alph*)]
  \item 
  \item 
\end{enumerate}


}

\section{Summary}
\label{sec:summary}


\begin{acknowledgements}
The author would like to acknowledge
\begin{enumerate}
  \item GOTM model with implemented wave
forcing from Prof. Alstair Jenkins.
  \item DARLA project and ONR ref. number
  \item \ldots 
\end{enumerate}

\end{acknowledgements}


\bibliographystyle{spbasic}
\bibliography{References_all}

\end{document}


\comments { 
 
%%%%%%%%%%%%% sample formula for latex

\bigskip{}

$\dfrac{g}{c}$

$k-\omega $   

$ \nicefrac{x}{2}$ 

$\left(F_{x,n}^{\mathrm{fric}},F_{y,n}^{\mathrm{fric}}\right)$


\begin{align*}
\mathcal{L}^{\mathrm{RS}} & =\alpha^{\mathrm{RS}}H_{\mathrm{s}},\\
\\\mathcal{L}^{\mathrm{ds,b}} & 
=\alpha^{\mathrm{ds,b}}\delta_{\mathrm{w}},\\
\\\mathcal{L}^{\mathrm{ds,s}} & 
=\alpha^{\mathrm{ds,s}}H_{\mathrm{s}}.\end{align*}

\begin{align}
F_{\alpha,n}^{\mathrm{ds}} & =\dfrac{k_{\alpha}g}{\left\Vert
\mathbf{k}\right\Vert
c}\left(f_{n}^{\mathrm{ds,b}}\left(z_{n}+D\right)S^{\mathrm{ds,b}}+f_{n}^{\mathrm{ds,s}}\left(\eta-z_{n}\right)
S^{\mathrm{ds,s}} \right)
\label{eq:wav-momentum}
\end{align}


\begin{table}
\noindent \begin{centering}
\footnotesize
\begin{tabular}{l l l l }
\hline 
 & $u_{\alpha,n}$ & $u_{\alpha,n}^{\mathrm{mass}}$ & 
\textrm{$F_{\alpha,n}^{\mathrm{wave}}$}
\tabularnewline
%\hline
\hline 
standard \\
GETM & $=u_{\alpha,n}$ & $=u_{\alpha,n}$ & $=0$\tabularnewline
%\hline 
radiation
stress & $=\overline{u}_{\alpha,n}^{\mathrm{L}}$ & $=\overline{u}_{\alpha,n}^{\mathrm{L}}$ & $=-\dfrac{\partial}{\partial x_{\beta}}\left\{ h_{n}\left\Vert \mathbf{k}\right\Vert E\left[\left(f^{\mathrm{cc}}f^{\mathrm{cs}}\right)_{n}\dfrac{k_{\alpha}k_{\beta}}{\left\Vert \mathbf{k}\right\Vert ^{2}}-\delta_{\alpha\beta}\left(\left(f^{\mathrm{ss}}f^{\mathrm{sc}}\right)_{n}-\dfrac{f_{n}^{\mathrm{RS}}\left(\eta-z_{n}\right)}{2\left\Vert \mathbf{k}\right\Vert h_{n}}\right)\right]\right\} $\tabularnewline
%\hline 
vortex
force & $=\overline{u}_{\alpha,n}^{\mathrm{E}}$ & $=\overline{u}_{\alpha,n}^{\mathrm{L}}$ & $=h_{n}\overline{u}_{\beta,n}^{\mathrm{Stokes}}\left(\dfrac{\partial\overline{u}_{\beta,n}^{\mathrm{E}}}{\partial x_{\alpha}}\right)_{z}-h_{n}\left(\dfrac{\partial}{\partial x_{\alpha}}\right)_{z}\left\{ \dfrac{\left\Vert \mathbf{k}\right\Vert E}{\sinh\left(2\left\Vert \mathbf{k}\right\Vert \left(\eta+D\right)\right)}\right\} +F_{\alpha,n}^{\mathrm{ds}}$\tabularnewline
\hline
\end{tabular}
\par\end{centering}

\caption{Juxtaposition of formulations\label{tab:methods}\protect \\
Summation is carried out over repeated indices with $\alpha,\beta\in\left\{ x,y\right\} $
and $x_{x}=x$, $x_{y}=y$, $u_{x}=u$, $u_{y}=v$. Please see sections
\ref{sec:vf_detail} and \ref{sec:rs_detail} for the description of specific
terms.}

\end{table}


%%%%%%%%%%%%%%%%%%%%%%%% END of Sample section % %%%%%%%%%%%%%%%%



















%%%% >>>>>>>>>>>>>>>>>>>>>>>>>>>>>>>>>>>>>>>>   cut from intro:


Dissipating gravity waves modifing turbulence quntities  


%Why this is important in shallow water\\


Modification of turbulence quantities inside water column enhances the vertical
mixing and directly affects hydrodynaimcs and sediment transport in the vicinity
of the breaking event which also modifyes beach evolution process. Several
studies show the importance of inclusion of wave generated turbulence and its
effects on sediment resuspention on increasing of total sediment transport in
surfzone \citep[e.g.][]{roelvink1989bar,yoon2010large}.


Turbulent Kenematic energy generate by breaking waves and shear production are
the main source of  

 The TKE production due to breaking waves could be orders of
magnitude larger than shear production close to the surface \citep{Terrayetal96}.  



Experiments:
The Wave breaking turbulence were studied in wave flumes in a controled
environents by many researchers \citep[e.g.][]{roelvink1989bar,yoon2010large}.
There are also field observations of wave turbulence reported
\citep[e.g.][]{goeorge et al, 1994; Trowbridge and elgar, 2001;}. On the
other hand limitations in operation of the equipments and controling of
environment conditions in nature, decreases the quality of these kind of field
measurments(\cite{yoon2010large}).

Acoustic Doppler Velocimetry (ADV) and Accoustic Doppler
Current Profilers (ADCP) have been extensively applied for turbulence
measurements in field and lab experiments. Coarse spatial resolution and
background noise plus high rate of aeration close to the surface and at the
trough level prevents the instrument to resolve much of the detail 
structures in the last 20 cm close to the surface \citep{wang2013free}.


\cite{yoon2010large} show similar order of magnitude between
$\bar{K}^{1/2}$ and $(\epsilon_{w}/\rho)^{1/3}$ from their large scale
flume experiment. Where :
\begin{align}
\bar{K}=\dfrac{1}{h} \int_{0}^{h} \bar{k}\, dx
\end{align}
and $\bar{k}$ is the turbulent kinetic energy (TKE). $\epsilon_{w}$ is total 
wave energy dissipation. They also stated that their quantitative comparisons
supports the hypothesis that wave energy dissipation is the main source of TKE
production in the surfzone\citep[e.g.][]{govender2002video}. From their results,
transition from spilling to plunging lessen the amount of transported TKE near
the bottom. 



This set of data:

Now the question is how we can incorporate the effects of surface breaking wave
in nearshore application. The majority of publication on the wave induced
turbulence and mixing were on upper ocean. 


Review on 1DV turbulence model reproducing wave effects:\\  

Main purpose:
The main goal of this paper is to provide a base for application and comparison
of methods for enhacing wave  breaking 


Here we aim for comparisons of TKE and EPS from diffrent models. All the wave
related forcing is implemented. We also take into account time depenednt tidal
elevation and velocities in orther to be fair when comparing production due to
current shear, bottom friction and surface wave induced turbulence.


assuming all wave dissipation enters to the water column at the same place. in
the other word there would be no replacing of the energy e.g. by rollers and so
on.


Litrature
Surface waves argued to be on of the main mechanisms of exchanging energy and
momntum between atmospher and ocean. In spite of many experimental and modeling
reseach have already done on wind, wave and current coupling mechanisms, the
lack of a robost conceptual model and physical explanation for interplay of
different phenomena in this area is still obvious.
Understanding and modeling of the turbulenec quantities in nearshre and surfzone
area is even less studied.



Rascle,2012





Observations of nearshore wave dissipation over muddy sea beds
Vertical structure of dissipation in the nearshore
The Effect of Wave Breaking on Surf-Zone Turbulence and Alongshore Currents: A Modeling Study
An application of the E-ε turbulence model for studying coupled air-sea boundary-layer structure
Modeling the influence of wave‐enhanced turbulence in a shallow tide‐and wind‐driven water column
Simulating the wave-enhanced layer under breaking surface waves with two-equation turbulence models
The influence of whitecapping waves on the vertical structure of turbulence in a shallow estuarine embayment
Structure and modeling of surf zone turbulence due to wave breaking
Near-surface wind-induced mixing in a mine lake

Title:

Modeling of the effects of the breaking waves on the turbulence structure in front of a tidal inlet.

Modeling of the turbulence structure enhancement due to active breaking waves.

On the modeling of wave-enhanced turbulence under surfzone breaking waves.

Modeling of active wave breaking turbulence enhancement   



TOC:

Introduction

Why this is important

Literature

Case study

        NRI

       

Observations

SWIFT data

Methods, models and governing equation

ROMS

SWAN

GOTM

   

Results

        final graphs of comparison of best craig and banner plus other methods

        How to choose which one is better

    ,m nxjk.bj iopbugeirfo;iaetu-aw t;FNM,.ARFGB9A   

Discussion

Compare amount of energy from dissipation should goes to water column

        Did not considering high frequency tial, how much this could affect the results?


categorize points based on amount of energy released

       

Conclusion


In this method calculation of surface flux and surface roughness includes
diffrent constants here we are discusing how these constants were choosen.

We desiged a broad range of sensivity analysis for both $\alpha^\mathrm{s}$ and $z_0^\mathrm{s}$.
Based on former researchs for the proposed range of these parametes in
litrature for $\alpha^\mathrm{s}$ values for 70 (jones,2008) $\sim$ 150 (prposed by
craig,1996 and Tray,1996) could be find. This parameter is based on wave age
parameters and for full developed system proposed arounf 100. Since for all
chosen points local wind >5.5 [ms-1] were chosen to be sure we are choosing the
right parameter we tested range of 70-200. 


 }

